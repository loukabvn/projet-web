\chapter{Introduction}

Le but de ce projet est de développer une plateforme de signalement pour la maintenance des
infrastructures d'une organisation. Cette plateforme permettra aux usagers de signaler une
anomalie sur une ressource simplement en scannant un QR code.

\chapter{Architecture}

\section{Modèle MVC}

\section{Frameworks}

\section{Sécurisation}

\chapter{Difficultés}

\chapter{Conclusion}

\chapter{Annexe}

\section{Notice d'utilisation}

L'installation peut se faire de deux manières différentes :

\begin{enumerate}
    \item Premièrement, copier l'archive \verb:install.tar.gz: sur la machine cible, ensuite
        l'extraire puis lancer le script \verb:install.sh: avec les droits administrateurs.

    \item Sinon l'application peut être installer directement depuis la machine cible en recupérant
        le script d'installation sur GitHub et en lançant le script comme precédemment :
\end{enumerate}

\begin{minted}{bash}
wget https://raw.githubusercontent.com/loukabvn/projet-web/main/install.sh
chmod u+x install.sh
su root
./install.sh
\end{minted}

Il faut ensuite renseigner les informations demandées c'est-à-dire les identifiants pour le
compte MySQL pour l'accès administrateur à la base de données et les identifiants pour le compte
administrateur de la plateforme.

Une fois l'installation terminée vous pouvez vous rendre sur :
\begin{center}
    \url{http://192.168.76.76:8080/ProjetWeb/home}
\end{center}
et vous connecter avec les identifiants renseignés.

\section{Jeu de données}

Un jeu de données pré-rempli est disponible pour effectuer des tests sur notre plateforme.
Pour pouvoir utiliser ces données il suffit d'exécuter avec mysql le script \verb:insertion.sql:.
Également, durant l'installation, l'insertion de ce jeu de données de test vous ait proposé, que
vous pouvez accepter ou refuser.

